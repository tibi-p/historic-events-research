% vim: set tw=78 sts=2 sw=2 ts=8 aw et ai:

Traditionally, historic events have been analyzed from the perspective of carefully selected texts, deemed historically significant. This has allowed historians to paint accurate descriptions of important events in our history, focusing on motives, narrative and their influence on human civilization.

However, as proven by the recent advent of the field of culturomics, introduced by \newcite{Michel14012011}, there is much to be learned by analyzing the large collections of books that are now also available in electronic format. This paper delves deeper into the implications of this approach for the analysis of historic events. Thus, the purpose is to devise a model that can be used for the automatic identification of important events in texts, by using quantitative data collected from millions of books instead of relying on understanding a few select texts. Moreover, we also want to semi-automatically link each important event to a small group of words that are very insightful for describing it.

The paper continues with a short overview of the work related to our research objectives. Section 3 introduces the historic events model that is built from the analyzed corpus and which consists of two important components: selecting the most descriptive features for each individual year and applying topic modelling to this model. In section 4, we present a selection of the results obtained using the hitoric events model. The paper ends with a discussion about future work and conclusions.
